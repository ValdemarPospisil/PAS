\documentclass{article}
\usepackage[czech]{babel}
\usepackage{amsmath}

\begin{document}

\section*{Poznámky k proměnným ve statistice}

\subsection*{Definice proměnné}
Proměnná je charakteristika, která může nabývat různých hodnot. V kontextu statistiky se proměnné dělí na několik typů:

\begin{itemize}
    \item \textbf{Kategorické proměnné} (nominal): Proměnné, které mají dvě nebo více kategorií, ale nemají přirozené pořadí. Například: barva očí (modrá, zelená, hnědá).
    \item \textbf{Ordinální proměnné} (ordinal): Proměnné, které mají dvě nebo více kategorií s přirozeným pořadím, ale rozdíly mezi kategoriemi nejsou nutně stejné. Například: hodnocení spokojenosti (nespokojen, neutrální, spokojen).
    \item \textbf{Intervalové proměnné} (interval): Proměnné, kde rozdíly mezi hodnotami jsou stejné, ale nemají přirozený nulový bod. Například: teplota ve stupních Celsia.
    \item \textbf{Poměrové proměnné} (ratio): Proměnné, které mají všechny vlastnosti intervalových proměnných a také přirozený nulový bod. Například: hmotnost, výška.
\end{itemize}

\subsection*{Popisné statistiky}
Pro popis proměnných se používají různé popisné statistiky:

\begin{itemize}
    \item \textbf{Průměr} (\(\bar{x}\)): Součet všech hodnot dělený počtem hodnot.
    \item \textbf{Medián}: Prostřední hodnota v seřazeném souboru dat.
    \item \textbf{Modus}: Nejčastěji se vyskytující hodnota v souboru dat.
    \item \textbf{Rozptyl} (\(s^2\)): Průměrná čtvercová odchylka hodnot od průměru.
    \item \textbf{Směrodatná odchylka} (\(s\)): Druhá odmocnina rozptylu.
\end{itemize}

\end{document}